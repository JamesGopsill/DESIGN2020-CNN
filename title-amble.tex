{
  \titlefont
  \small
  \begin{wrapfigure}{r}{0.2\textwidth}
    \raggedleft
    \vspace{-0.2cm}
    \includegraphics[width=1.8cm]{figs/design-logo.png}
  \end{wrapfigure}
  \noindent \textbf{INTERNATIONAL DESIGN CONFERENCE - DESIGN 2020}\\
  \url{https://doi.org/10.1017/dsd.2020.93}

  \vspace{2cm}

  \Large\noindent\textbf{DEMOCRATISING DESIGN THROUGH SURROGATE MODEL NEURAL NETWORKS OF COMPUTER AIDED DESIGN REPOSITORIES}

  \vspace{1cm}

  \normalsize \noindent J.\ Gopsill and S.\ Jennings \\[0.2cm]
  \footnotesize \noindent University of Bath, United Kingdom \\[0.1cm]
  \footnotesize \noindent J.A.Gopsill@bath.ac.uk\\

  \begin{mdframed}[backgroundcolor=gray!20] 
    \normalsize \noindent \textbf{Abstract} \\
    \normalfont The capability to manufacture at home is continually increasing with technologies, such as 3D printing. 
    However, the ability to design products suitable for manufacture and use remains a highly-skilled and knowledge intensive activity.
    This has led to `content creators' providing vast repositories of manufacturable products for society, however challenges remain in the search \& retrieval of models.
    This paper presents the surrogate model convolutional neural networks approach to search and retrieve CAD models by mapping them directly to their real-world photographed counterparts.
  \end{mdframed}

  \small \noindent \textit{Keywords: 3D printing, computer-aided design (CAD), design informatics, surrogate models,\\ convolutional neural networks}

  \vspace{0cm}
}
